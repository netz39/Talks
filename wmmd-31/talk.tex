%%%%%%%%%%%%%%%%%%%%%%%%%%%%%%%%%%%%%%%%%%%%%%%%%%%%%%%%%%%%%%%%%%%%%%%%
%%% documentclass and packages
%%%%%%%%%%%%%%%%%%%%%%%%%%%%%%%%%%%%%%%%%%%%%%%%%%%%%%%%%%%%%%%%%%%%%%%%
%\RequirePackage{atbegshi}           % workaround for newer PGF versions
%\documentclass[hyperref={pdfpagelabels=false}]{beamer}
\documentclass{beamer}
% https://sourceforge.net/tracker/index.php?func=detail&aid=1848912&group_id=92412&atid=600660
\usepackage{lmodern}
\usepackage[T1]{fontenc}
\usepackage[utf8]{inputenc}
\usepackage{textcomp}
\usepackage[ngerman]{babel}
\usepackage[babel,english=american,german=guillemets]{csquotes}	% french
%\usepackage{microtype}

% colors for listings
\definecolor{lightergray}{gray}{.95}
\definecolor{darkblue}{rgb}{0,0,0.5}
\definecolor{darkgreen}{rgb}{0,0.5,0}
\definecolor{darkred}{rgb}{0.5,0,0}
\definecolor{darkerblue}{rgb}{0,0,0.4}
\definecolor{darkergreen}{rgb}{0,0.4,0}
\definecolor{darkerred}{rgb}{0.4,0,0}

\usepackage{listings}
\lstloadlanguages{HTML,XML}
\lstset{
    basicstyle=\ttfamily\small\mdseries,
    keywordstyle=\bfseries\color{darkblue},
    identifierstyle=,
    commentstyle=\color{darkgray},
    stringstyle=\itshape\color{darkred},
    frame=none,
    showstringspaces=false,
    tabsize=4,
    backgroundcolor=\color{lightergray},
}

%%%%%%%%%%%%%%%%%%%%%%%%%%%%%%%%%%%%%%%%%%%%%%%%%%%%%%%%%%%%%%%%%%%%%%%%
%%% preparations for beamer
%%%%%%%%%%%%%%%%%%%%%%%%%%%%%%%%%%%%%%%%%%%%%%%%%%%%%%%%%%%%%%%%%%%%%%%%
\useinnertheme{default}
\useoutertheme{infolines}
%\usecolortheme[rgb={0.28,0.37,0.52}]{structure}
\usecolortheme[rgb={0.18,0.23,0.33}]{structure}
%\usecolortheme{beaver}
\usefonttheme{structurebold}

%%% Ränder vergrößern für's Café Central
\setbeamersize{text margin left=1.2cm}
\setbeamersize{text margin right=1.2cm}

%%%%%%%%%%%%%%%%%%%%%%%%%%%%%%%%%%%%%%%%%%%%%%%%%%%%%%%%%%%%%%%%%%%%%%%%
%%% images
%%%%%%%%%%%%%%%%%%%%%%%%%%%%%%%%%%%%%%%%%%%%%%%%%%%%%%%%%%%%%%%%%%%%%%%%
\pgfdeclareimage[height=0.75\paperheight]{strasse}{strasse}
\pgfdeclareimage[height=0.75\paperheight]{arbeitsraum}{arbeitsraum}
\pgfdeclareimage[height=0.75\paperheight]{lounge}{lounge}
\pgfdeclareimage[height=0.75\paperheight]{mate}{mate}
\pgfdeclareimage[height=0.75\paperheight]{regal}{regal}
\pgfdeclareimage[height=0.75\paperheight]{werkstatt}{werkstatt}

%%%%%%%%%%%%%%%%%%%%%%%%%%%%%%%%%%%%%%%%%%%%%%%%%%%%%%%%%%%%%%%%%%%%%%%%
%%% title, author, date
%%%%%%%%%%%%%%%%%%%%%%%%%%%%%%%%%%%%%%%%%%%%%%%%%%%%%%%%%%%%%%%%%%%%%%%%
\title[Netz39 e.\,V.]{Netz39 e.\,V.}
\subtitle{Leibnizstraße 32 -- die ersten Monate}
\author{Alexander Dahl}
\institute[netz39.de]{\url{http://www.netz39.de/}}
\date{2012-10-08}
\subject{subj}
\keywords{foo, bar}

%%%%%%%%%%%%%%%%%%%%%%%%%%%%%%%%%%%%%%%%%%%%%%%%%%%%%%%%%%%%%%%%%%%%%%%%
%%% document
%%%%%%%%%%%%%%%%%%%%%%%%%%%%%%%%%%%%%%%%%%%%%%%%%%%%%%%%%%%%%%%%%%%%%%%%
\begin{document}

\begin{frame}
	\titlepage
\end{frame}

%\begin{frame}{Überblick}
    %\tableofcontents
%\end{frame}

\section{Hackerspaces}

\subsection{Definition}

\begin{frame}{Hackerspaces}
    \begin{block}{hackerspaces.org:}
        \begin{quote}
            Hackerspaces are community-operated physical places, where
            people can meet and work on their projects.
        \end{quote}
    \end{block}
    \pause
    \begin{block}{de.wikipedia.org:}
        \begin{quote}
            Ein Hackerspace (von Hacker und Space, engl. für Raum) oder
            Hackspace ist ein physischer, häufig offener Raum, in dem
            sich Hacker und Interessierte treffen und austauschen
            können. Mitglieder mit Interessen an Wissenschaft,
            Technologie und digitaler Kunst organisieren sich meist in
            Vereinen.
        \end{quote}
    \end{block}
    \pause
    \begin{block}{29. Webmontag}
        \url{https://bitbucket.org/netz39/talks/downloads}
    \end{block}
\end{frame}

\subsection{Magdeburg}

\begin{frame}{Leibnizstraße 32}
    \begin{figure}
        \pgfuseimage{strasse}
    \end{figure}
\end{frame}

\begin{frame}{Was wollen wir?}
    \begin{block}{Über uns}
        \begin{quote}
            Netz39 versteht sich als Anlaufpunkt für den technischen,
            gesellschaftlichen und kulturellen Austausch im Bereich
            informationsverarbeitender Technologien …
        \end{quote}
    \end{block}
    \begin{block}{Und wirklich?}
        \begin{itemize}
            \item Elektronik
            \item Software
            \item Werkzeug
            \item Vorträge/Workshops
            \item …
        \end{itemize}
    \end{block}
\end{frame}

\section{Verein}

\subsection{Zuletzt}

\begin{frame}[label=secbaellebad]{Was zuletzt geschah …}
    \begin{block}{Beteiligung}<1->
        \begin{itemize}
            \item neue Mitglieder willkommen
            \item Mitgliedsantrag online
            \item Mitgliedsbeitrag pro Monat: 30 € oder 8 €
            \pause
            \item Fördermitgliedschaft
            \item Spenden
        \end{itemize}
    \end{block}
    \begin{block}{Angebot}<3->
        \begin{itemize}
            \item Raum
            \item Leute
            \item Know-How
        \end{itemize}
    \end{block}
\end{frame}

\subsection{Aktueller Stand}

\begin{frame}{Was haben wir?}
    \begin{itemize}
        \item Platz: 100m²
        \pause
        \item Küche
        \item Sofas
        \item Tische
        \pause
        \item Internetzugang
        \item Club-Mate
        \pause
        \item Unordnung
        \item ToDo-Liste
    \end{itemize}
\end{frame}

\begin{frame}{Lounge}
    \begin{figure}
        \pgfuseimage{lounge}
    \end{figure}
\end{frame}

\begin{frame}{Arbeitsraum}
    \begin{figure}
        \pgfuseimage{arbeitsraum}
    \end{figure}
\end{frame}

\begin{frame}{Was machen wir gerade?}
    \begin{itemize}
        \item Regal
        \pause
        \item Rollladensteuerung
        \pause
        \item Schließanlage
        \pause
        \item Versicherung
        \item Feuerlöscher
        \item Beschallung
    \end{itemize}
\end{frame}

\begin{frame}{Bücherregal}
    \begin{figure}
        \pgfuseimage{regal}
    \end{figure}
\end{frame}

\begin{frame}{Was haben wir vor?}
    \begin{itemize}
        \item Freifunk
        \item Illumination
        \item Dance Dance Revolution
        \item Twitterwall
        \item Mikrocontroller-Basteleien
        \item Lötarbeitsplätze
        \item 3D-Drucker
        \item Pfandkisten bauen
    \end{itemize}
\end{frame}

\begin{frame}{Getränkeautomat}
    \begin{figure}
        \pgfuseimage{mate}
    \end{figure}
\end{frame}

\begin{frame}{Werkstatt}
    \begin{figure}
        \pgfuseimage{werkstatt}
    \end{figure}
\end{frame}

\section{Kontakt}

\begin{frame}{Kontakt}
    \begin{center}
        \begin{description}[Twitter/identi.ca]
            \item[WWW] \url{http://www.netz39.de/}
            \item[Twitter/identi.ca] @netz39
            \item[E-Mail] kontakt@netz39.de
            \item[Mailingliste] list@netz39.de
            \item[IRC] \#netz39 auf freenode
        \end{description}
    \end{center}
\end{frame}

\appendix

\section{Lizenz}

\begin{frame}{Lizenz}
    \begin{block}{Bilder}
        Die Bilder sind unter folgender Creative Commons-Lizenz
        veröffentlicht: \emph{Namensnennung-Keine kommerzielle
        Nutzung-Weitergabe unter gleichen Bedingungen 3.0}, (CC-BY-NC-SA
        3.0). Siehe auch:
        \url{http://www.lespocky.de/fotos/opensource/netz39/}
    \end{block}
    \begin{block}{Folien}
        Die Folien sind freigegeben unter \emph{Creative Commons
        Namensnennung-Weitergabe unter gleichen Bedingungen 3.0 Deutschland
        Lizenz.} (CC-BY-SA 3.0). Download unter:
        \url{https://bitbucket.org/netz39/talks/downloads}
    \end{block}
\end{frame}

\end{document}
